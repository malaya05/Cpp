\documentclass[11pt]{article}
\usepackage[margin=1in]{geometry}
\usepackage{hyperref}
\usepackage[explicit]{titlesec}
\usepackage[table,dvipsnames]{xcolor}
%\usepackage{times}
\usepackage{amssymb}
\usepackage{amsmath}
\usepackage{framed}
\usepackage{fancyhdr}
\pagestyle{fancy}
% Header Specifications
\lhead{\Large A257}
\chead{}
\rhead{ }
% Footer Specifics
\lfoot{}
\cfoot{\thepage}
\rfoot{}
\renewcommand{\headrulewidth}{0.4pt}
\renewcommand{\footrulewidth}{0.4pt}
\setlength{\headheight}{26pt}
\setlength{\parindent}{0pt}

\begin{document}

%-----------
\section*{\underline{Generating Sentences}}
For this lab, you will build an English sentence generator.  That is,
you will write a program that will be able to produce random English
sentences, and English sentences containing specific words.

\section*{Some Background}
Natural Language Processing (NLP) deals with making computers
understand what people mean when they speak or write.  This is a much
harder problem than it might seem. Some of the challenges include
ambiguity, the use of idioms, and the use of sarcasm.  Such sentences
as ``Will Will will the will to Will?'' are difficult for people to
understand, much less a computer.  People are much better at figuring
out meaning based on context clues.  For example, take the word
``bank''.  You could say ``I need to deposit money in the bank.''  You
could also say ``The men are fishing from the bank of the river.''
Needless to say, NLP has had to overcome numerous challenges, and they
still aren't all solved!\\ 

For your lab, you will create a program to generate English sentences.
We aren't concerned with understanding the sentence.  Generating a
sentence (in any language) that makes sense in context is also a hard
problem.  Ever talked to a chat-bot?  It's very hard for a computer to
grasp the \textit{meaning} of a sentence.  For example, say you have
the sentence ``That movie was great.  I was able to catch up on the
sleep I was missing.''  A computer is likely to think that this
sentence means the person actually liked the movie since it contains
the word ``great''.  However, we can see that the person didn't like
the movie at all since he/she fell asleep during the movie.\\

For this lab, you'll just generate random sentences for fun.\\


\section*{Program Requirements}
A grammar is given below that describes several syntactically correct
forms of English sentences.  You will use this grammar to generate
sentences.  Your program should be able to generate random sentences that are grammatically
    correct. They most likely will not make sense, however.

\newpage
\subsection*{The Grammar}
S: Sentence\\
NP: Noun Phrase\\
VP: Verb Phrase\\
PP: Prepositional Phrase (on the couch, in the class, over the hill, ...)\\
ART: Article (a, the, an, ...) \\
NOUN: Noun (boy, dog, rug ..) (person, place, thing, or idea)\\
VERB: Verb (run, walk, swim, is, are, were ...)(action or state of being words)\\
ADJ: Adjective (purple, fuzzy, sharp, fast ...) (words that describe nouns)\\
ADV: Adverb (lazily, quickly, slowly ...) (words that describe verbs, often end in ``-ly'')\\
PREP: Preposition (on, over, through, in, around ... ) (words that fit in the sentence ``The bunny went \textit{PREP} the log'')\\

\begin{align*}
<S> &::= <NP><VP>\\
<NP> &::= <NOUN> | <ART><NOUN> | <NP><PP>\\
<VP>  &::= <VERB> | <VP><NP> | <VP><ADJ> | <VP><ADV>\\
<PP> &::= <PREP><NP>
\end{align*}

\subsection*{Generating a Random Sentence}
Generate a sentence that follows the grammar by picking random
substitutions from the grammar and fill in with random words from the
word lists.  I have provided lists of Nouns, Verbs, Adjectives,
Adverbs, Articles, and Prepositions.  You may modify these lists if
you want, but keep the words safe for work!  For example, one sentence
could be\\

\noindent S\\
NP VP\\
NP PP VP\\
NP PP VERB\\
ART NOUN PP VERB\\
ART NOUN PREP NP VERB\\
ART NOUN PREP ART NOUN VERB\\
The bunny in the cage slept.\\


\newpage
\section*{Testing}
Have your program generate 4 random sentences. The sentences should
have variations in them (i.e.  you can't pick all of the NOUN VERB
sentences).  For each sentence, show that the sentence is a valid sentence. \textbf{Turn in a
  pdf of screen shots of the sentences and the derivations.} You may use derivation trees.

\subsection*{Example}
Sentence: a water prefer many research like place near history behind the night\\
Words given: None\\
Production:\\
(S)\\
(NP) \textcolor{red}{(VP)}\\
(NP) \textcolor{red}{(VP) (NP)}\\
(NP) (VP) (NP) \textcolor{red}{(PP)}\\
(NP) (VP) (NP) \textcolor{red}{(PREP) (NP)}\\
(NP) (VP) \textcolor{red}{(NP)} (PP) (PREP) (NP)\\
(NP) (VP) \textcolor{red}{(NP) (PREP)} (NP) (PREP) (NP)\\
(NP) \textcolor{red}{(VP)} (NP) (PREP) (NP) (PREP) (NP) (PREP) (NP)\\
(NP) \textcolor{red}{(VP) (ADV)} (NP) (PREP) (NP) (PREP) (NP) (PREP) (NP)\\
\textcolor{red}{(NP)} (VP) (ADV) (NP) (PREP) (NP) (PREP) (NP) (PREP) (NP)\\
\textcolor{red}{(ART)(NOUN)} (VP) (ADV) (NP) (PREP) (NP) (PREP) (NP) (PREP) (NP)\\
(ART)(NOUN)\textcolor{red}{ (VP)} (ADV) (NP) (PREP) (NP) (PREP) (NP) (PREP) (NP)\\
(ART)(NOUN) \textcolor{red}{(VERB) (ADV)} (NP) (PREP) (NP) (PREP) (NP) (PREP) (NP)\\
(ART)(NOUN) (VERB) (ADV)\textcolor{red}{ (NP)} (PREP) (NP) (PREP) (NP) (PREP) (NP)\\
(ART)(NOUN) (VERB) (ADV) \textcolor{red}{(NOUN)} (PREP) (NP) (PREP) (NP) (PREP) (NP)\\
(ART)(NOUN) (VERB) (ADV) (NOUN) (PREP) \textcolor{red}{(NP)} (PREP) (NP) (PREP) (NP)\\
(ART)(NOUN) (VERB) (ADV) (NOUN) (PREP) \textcolor{red}{(NOUN)} (PREP) (NP) (PREP) (NP)\\
(ART)(NOUN) (VERB) (ADV) (NOUN) (PREP) (NOUN) (PREP) \textcolor{red}{(NP)} (PREP) (NP)\\
(ART)(NOUN) (VERB) (ADV) (NOUN) (PREP) (NOUN) (PREP) \textcolor{red}{(NOUN)} (PREP) (NP)\\
(ART)(NOUN) (VERB) (ADV) (NOUN) (PREP) (NOUN) (PREP) (NOUN) (PREP) \textcolor{red}{(NP)}\\
(ART)(NOUN)(VERB)(ADV)(NOUN)(PREP)(NOUN)(PREP)(NOUN)(PREP)\textcolor{red}{(ART)(NOUN)}\\
a      water  prefer many  research  like   place   near  history   behind  the    night\\

\hfill\\ \textcolor{red}{\textbf{You should use color or italics or bold to show which
  non-terminal you are replacing at each step}}

\newpage
\section*{Hints and Tips}
\begin{itemize}
\item
  I have given you a Dictionary class that reads in the lists of words
  and creates vectors containing each part of speech. You can then get
  random words.  You don't have to use it, but it'll probably make your
  life easier.  You may also modify this class.
\item
  You may assume that the words the user gives you are actually the
  part of speech they claim.  For instance, if you ask for a noun, you
  may assume that the user gives you a noun.
\item
  I suggest having at least the following functions.  The more you
  break up your code, the easier it will be to piece everything
  together.  Recursion is also your friend!!!
  \begin{itemize}
  \item
    string generateNounPhrase()
  \item
    string generateVerbPhrase()
  \item
    string generatePrepPhrase()
  \end{itemize}
  
\end{itemize}


%------
\section*{Check list}
Your code must have a name header as described in the syllabus.
Each of these items is worth two points of your grade.

Your output must
\begin{itemize}
\item
  Have a title in the output
\item 
  Include a short paragraph indicating the purpose of the
  application to the user
\item 
  Use complete and clear sentences -- do not assume the user knows
  what you expect from him/her.
\item 
  Have NO grammatical errors and/or spelling errors.
\item 
  Have NO typos -- this includes using capital letters at the
  beginning of a sentence and punctuation at the end of sentences.
\end{itemize}

\newpage
%-----------------
\section*{Example Output}
Here are some examples for what your program's output could look
like. Yours doesn't have to look exactly like this. User input is in
{\color{blue} blue}.  The user's words are in all capital letters to
make them stand out in the generated sentence.

\ttfamily
\hfill\\
Title Goes Here\\
----------------------------\\
Purpose Goes Here\\
---------------------------\\
\\
Choose an option: \\
1. Generate Random Sentence\\
2. Quit\\
{\color{blue} 1}\\
\\
guy copy education along a side after a case\\
\\
Choose an option: \\
1. Generate Random Sentence\\
2. Quit\\
{\color{blue} 1}\\
\\
the name tell financial east an system same\\
\\
Choose an option: \\
1. Generate Random Sentence\\
2. Quit\\
{\color{blue} 1}\\
\\
the government concerning a office anti a eye down the idea cry unusual little\\
\\
Choose an option: \\
1. Generate Random Sentence\\
2. Quit\\
{\color{blue} 3}\\
\rmfamily

%------
\section*{What to Turn In}
Upload your .cpp and .h (if you have any) files and the pdf of your
output and derivations as described in the program requirements to Canvas.


\end{document}
